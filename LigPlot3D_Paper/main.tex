%%%%%%%%%%%%%%%%%%%%%%%%%%%%%%%%%%%%%%%%%%%%%%%%%%%%%%%%%%%%%%%%%%%%%
%% This is a model paper for LigPlot3D using the achemso class
%%%%%%%%%%%%%%%%%%%%%%%%%%%%%%%%%%%%%%%%%%%%%%%%%%%%%%%%%%%%%%%%%%%%%
\documentclass[journal=jacsat,manuscript=article]{achemso}

\usepackage[version=3]{mhchem} % Formula subscripts using \ce{}
\usepackage{graphicx}
\usepackage{float}
\usepackage{xspace}
\usepackage{caption}
\usepackage{subcaption}
\usepackage[dvipsnames]{xcolor}
\usepackage{hyperref}

\newcommand*\mycommand[1]{\texttt{\emph{#1}}}

\author{Nitin Kumar Singh}
\affiliation[Indian Institute of Technology (IIT) Gandhinagar]
{Discipline of Chemical Engineering, Indian Institute of Technology (IIT) Gandhinagar, Palaj, Gujarat 382355, India}
\email{singh_nitin@iitgn.ac.in}

\title[LigPlot3D]
  {LigPlot3D: An Interactive Web-Based Framework for Geometric Analysis of Protein-Ligand Interactions}

\abbreviations{PDB, NGL, GUI, API}
\keywords{Molecular Docking, Visualization, Web Application, Interaction Analysis}

\begin{document}

\begin{abstract}
Molecular docking studies routinely generate multiple candidate binding poses, creating a practical need for rapid and interpretable inspection of protein--ligand interactions beyond numerical scoring functions. While established tools provide comprehensive interaction profiling or schematic visualization, they often require local installation or produce static outputs that limit interactive exploration. Here, I present \textit{LigPlot3D}, a web-based application for interactive, client-side analysis of protein--ligand complexes. LigPlot3D combines browser-native three-dimensional molecular rendering with deterministic, geometry-based identification of key non-covalent interactions, including hydrogen bonds, salt bridges, hydrophobic contacts, $\pi$-stacking, and halogen bonds. All analysis is performed locally within the browser, ensuring data privacy and eliminating server-side dependencies. Using representative docking complexes, LigPlot3D enables rapid qualitative assessment of binding modes and interaction geometry through an intuitive interface that integrates interactive visualization with structured interaction tables. By emphasizing accessibility, interactivity, and reproducible geometric criteria, LigPlot3D complements existing interaction profilers and visualization tools and supports efficient post-docking analysis in early-stage structure-based workflows.
\end{abstract}


\section{Introduction}
Molecular docking remains a widely used computational approach for predicting plausible binding modes and approximate affinities of small molecules in protein binding sites, supporting structure-based drug discovery workflows and virtual screening pipelines \cite{trott2010autodockvina}. In practice, docking campaigns typically generate multiple poses per ligand (and often multiple ligands per target), creating an immediate downstream need for rapid, reliable inspection of poses, ranking consistency, and the chemical plausibility of predicted binding modes. Beyond numerical scores, the interpretability of docking results often depends on verifying whether a pose reproduces key physicochemical determinants of binding, such as hydrogen-bond patterns and local contact geometry in the binding site.

A common strategy for increasing interpretability is the explicit identification of non-covalent interactions between ligand and receptor. Automated interaction profilers such as the Protein--Ligand Interaction Profiler (PLIP) provide systematic detection and reporting of interaction types in macromolecular structures, enabling consistent characterization of binding determinants across complexes \cite{salentin2015plip}. Recent advances have extended PLIP to incorporate protein--protein interactions, reflecting the broader demand for scalable interface analysis as structural datasets and predicted complexes continue to grow \cite{schake2025plip}. These profilers offer robust interaction taxonomies and batch-oriented workflows, but they are not primarily designed as lightweight, web-first environments for rapid, interactive inspection of docking results across multiple poses and ligands.

Complementary to interaction profiling, schematic visualization tools such as LigPlot+ generate 2D interaction diagrams that summarize hydrogen bonds and non-bonded contacts, and support comparative analysis across related ligand--protein complexes \cite{laskowski2011ligplotplus}. While 2D schematics can be effective for communicating interaction hypotheses and for comparing congeneric series, docking analysis frequently benefits from interactive exploration of alternative poses, immediate cross-comparison of pose-specific interactions, and streamlined access without local installation---particularly in early-stage triage and iterative design settings.

To address these practical needs, I present \textit{LigPlot3D}, a web-based application for interactive analysis of docking outputs and protein--ligand complexes. LigPlot3D is designed to support rapid pose-centric inspection by combining structure visualization with geometry-based identification of hydrogen bonds and contact-level summaries, enabling users to compare alternative poses and ligands within a unified, accessible interface. By emphasizing lightweight deployment and interactive workflows, LigPlot3D aims to complement established profilers and diagram generators by improving the usability of docking post-processing and facilitating reproducible, inspection-driven decision-making.

\section{Model and Methods}

\subsection{Architectural Framework}
LigPlot3D is implemented as a fully client-side Single Page Application (SPA), designed to prioritize interactive performance, data locality, and platform independence. All structure parsing, interaction analysis, and visualization are executed entirely within the user’s web browser, eliminating server-side dependencies and preventing external transmission of structural data.

The application is developed using \textbf{React.js} (v19.2) \cite{reactjs} with a functional component architecture. React hooks are employed for deterministic state management of molecular representations, interaction tables, and user-driven selections. The build and bundling pipeline is based on \textbf{Vite} (v7.2), which enables efficient module resolution and fast incremental rebuilds during development. Layout, typography, and responsive styling are handled using \textbf{TailwindCSS}, ensuring consistent rendering across display sizes and devices.

\subsection{Molecular Visualization Engine}
Three-dimensional molecular visualization is performed using a hybrid WebGL-based rendering layer that integrates \textbf{NGL Viewer} (v2.4.0) \cite{rose2018ngl} and \textbf{3Dmol.js} (v2.5.3) \cite{rego20153dmol}. NGL Viewer is used for rendering macromolecular structures, including protein cartoons, surfaces, and backbone representations, and provides robust support for standard structural formats such as PDB and mmCIF. In parallel, 3Dmol.js is employed for efficient rendering of small molecules and localized interaction geometries.

A custom synchronization layer coordinates camera state, orientation, and selection events between the two rendering engines, allowing seamless transitions between global protein views and ligand-focused inspection while preserving spatial context.

\subsection{Geometric Interaction Algorithms}
LigPlot3D incorporates a deterministic, geometry-based interaction detection module intended for rapid, pose-centric analysis of protein--ligand complexes. Interaction classification is performed using explicit distance- and angle-based criteria derived from commonly adopted conventions in structural biology. All interaction calculations are executed client-side using atomic coordinates parsed directly from the loaded structure files.

Geometric thresholds and atom classifications are defined centrally within the application configuration (\texttt{constants.ts}), ensuring transparency and reproducibility of interaction assignments. All active thresholds are documented within the application interface, allowing users to inspect the criteria underlying interaction detection.

\subsubsection{Hydrogen Bonds}
Hydrogen bonds are identified between potential donor and acceptor atoms restricted to nitrogen, oxygen, and sulfur. A hydrogen bond is assigned when the donor–acceptor heavy-atom separation is less than or equal to 3.5~\AA\ and the donor–hydrogen–acceptor angle is greater than or equal to $120^{\circ}$, enforcing geometric directionality. Fluorine atoms are explicitly excluded from the acceptor set due to their limited hydrogen-bonding capacity in biological environments.

\subsubsection{Salt Bridges}
Salt bridges are defined as electrostatic interactions between oppositely charged ionizable groups. Positively charged centers include nitrogen atoms in the side chains of arginine (NH1, NH2), lysine (NZ), and histidine (ND1, NE2), while negatively charged centers correspond to carboxylate oxygen atoms in aspartate (OD1, OD2) and glutamate (OE1, OE2). A salt bridge is assigned when the distance between charge centers is less than or equal to 4.0~\AA.

\subsubsection{Hydrophobic Contacts}
Hydrophobic contacts are detected between non-polar carbon atoms, including both aliphatic and aromatic carbons. A distance cutoff of 4.5~\AA\ is applied to capture van der Waals contacts that contribute to hydrophobic stabilization within the binding site.

\subsubsection{$\pi$-Stacking Interactions}
Aromatic interactions are evaluated for residues containing planar ring systems, including phenylalanine, tyrosine, tryptophan, and histidine. Ring centroids and plane normals are computed to classify interaction geometry. Interactions are assigned when the centroid–centroid distance is less than or equal to 5.5~\AA. Parallel stacking interactions require an inter-planar angle of $30^{\circ}$ or less with a lateral offset not exceeding 2.0~\AA, while T-shaped stacking interactions are identified by inter-planar angles greater than or equal to $60^{\circ}$.

\subsubsection{Halogen Bonds}
Halogen bonds are identified between halogen donors (Cl, Br, I) and nucleophilic acceptors (N, O, S). An interaction is assigned when the halogen–acceptor separation is less than or equal to 3.5~\AA\ and the C–X$\cdots$acceptor angle is greater than or equal to $140^{\circ}$, consistent with sigma-hole directionality.

\subsubsection{Metal Coordination}
Metal coordination interactions are detected for common biologically relevant metals, including Zn, Fe, Mg, and Ca, interacting with electronegative atoms (N, O, S). A stringent distance threshold of 2.8~\AA\ is applied to identify coordination contacts. The current implementation relies exclusively on geometric criteria and does not incorporate energetic scoring or solvent models.


\section{Results and Discussion}

\subsection{Interactive Exploration of Binding Interfaces}
LigPlot3D provides an integrated interface composed of a central three-dimensional visualization viewport and a configurable analytical sidebar. Upon loading a PDB structure, the application parses the coordinate file, identifies ligand entities, and constructs a localized representation of the binding environment. Users may selectively control protein representations, ligand styles, and interaction overlays through explicit interface controls.

Figure~\ref{fig:initial} illustrates the initial state of the application, prioritizing a clean interface for file upload. The landing page allows users to drag-and-drop structure files directly into the browser window, as shown in Figure~\ref{fig:initial}. Once a structure is loaded, LigPlot3D enables interactive inspection of key non-covalent interactions. An example interaction analysis for the JAK2 complex is shown in Figure~\ref{fig:loaded}. Users can dynamically toggle visibility for hydrogen bonds, salt bridges, hydrophobic contacts, $\pi$-stacking, and halogen bonds through the sidebar controls, allowing for focused analysis of specific interaction types.

\subsection{Interaction Detection and Geometric Consistency}
The interaction detection module was evaluated by examining representative protein--ligand complexes commonly used in docking benchmarks. For the Macrophage Migration Inhibitory Factor (MIF) complex\cite{crichlow2009structural}, LigPlot3D identified canonical active site interactions consistent with structural reporting (Figure~\ref{fig:loaded}). The detected interactions demonstrated qualitative agreement with reference profiles, reflecting the deterministic nature of the geometry-based criteria.

The application applies strict distance and angular constraints for interaction assignment, which reduces the inclusion of weak or geometrically ambiguous contacts. In practice, this results in concise interaction maps that emphasize structurally meaningful interactions rather than exhaustive contact enumeration. While LigPlot3D does not aim to provide an exhaustive interaction taxonomy, the implemented interaction classes capture the dominant non-covalent features relevant for docking pose assessment.

\subsection{Performance and Accessibility}
All visualization and interaction analysis in LigPlot3D is executed client-side within a standard web browser. In routine use, the application loads typical protein--ligand complexes without perceptible delay and maintains smooth interactive frame rates for structures containing up to approximately $10^5$ atoms. Because no server-side processing is required, LigPlot3D can be used in restricted or offline environments and does not transmit structural data externally.

This deployment model lowers the barrier to entry for docking result inspection and facilitates rapid review of binding hypotheses, making the tool suitable for early-stage pose triage, educational use, and collaborative discussion settings.

\begin{figure}[H]
    \centering
    \includegraphics[width=0.95\textwidth]{Figures/initial_view.png}
    \caption{The clean, minimal interface of LigPlot3D. Users can upload a PDB file or input a PDB ID to instantly load the structure.}
    \label{fig:initial}
\end{figure}

\begin{figure}[H]
    \centering
    \includegraphics[width=0.95\textwidth]{Figures/loaded_view.png}
    \caption{Interactive analysis of the Macrophage Migration Inhibitory Factor (MIF) complex\cite{crichlow2009structural}. LigPlot3D detects and visualizes hydrogen bonds (dashed lines), hydrophobic contacts, and $\pi$-stacking interactions in 3D. The sidebar provides granular control over visualization parameters and a detailed interaction table.}
    \label{fig:loaded}
\end{figure}

\subsection{Comparison with Existing Tools}
To contextualize LigPlot3D within the landscape of interaction analysis and visualization tools, I compared its functionality with LigPlot+\cite{laskowski2011ligplotplus} and the Protein--Ligand Interaction Profiler (PLIP)\cite{salentin2015plip,schake2025plip}.

LigPlot+ generates publication-quality two-dimensional interaction schematics by projecting three-dimensional complexes into flattened diagrams. These representations are effective for static reporting and comparative summaries but inherently sacrifice spatial depth information. LigPlot3D differs in that it preserves the full three-dimensional geometry of the binding site, enabling users to interrogate interaction orientations and steric relationships interactively within the browser.

PLIP provides a comprehensive and well-validated interaction profiling framework, including advanced interaction classes such as water bridges and metal coordination, and supports both protein--ligand and protein--protein interfaces. However, PLIP primarily operates through server-based reports or command-line workflows that generate static outputs or external visualization scripts. LigPlot3D complements this functionality by offering an interactive, client-side environment focused on rapid docking pose inspection and immediate visual feedback, without requiring data upload or local software installation.

\section{Conclusion}
In this study, I presented LigPlot3D, a web-based application for interactive analysis of protein--ligand binding interfaces. LigPlot3D integrates browser-native three-dimensional molecular visualization with deterministic, geometry-based detection of key non-covalent interactions, enabling rapid qualitative assessment of docking poses and binding modes.

By performing all parsing, visualization, and interaction analysis client-side, LigPlot3D eliminates the need for local software installation or remote server execution, while preserving data privacy. The application is designed to complement established interaction profilers and schematic visualization tools by emphasizing interactivity, accessibility, and immediate structural feedback within a lightweight workflow.

LigPlot3D is particularly well suited for early-stage docking analysis, educational use, and rapid inspection of binding hypotheses where intuitive exploration of interaction geometry is essential. Future development will focus on extending supported interaction annotations, incorporating optional energy-related descriptors, and enabling analysis of time-resolved structural data such as molecular dynamics trajectories.

\section{Data and Software Availability}
LigPlot3D is open-source software released under the MIT License.
\begin{itemize}
    \item \textbf{Web Application:} \href{https://singhnitink.github.io/LigPlot3D/}{https://singhnitink.github.io/LigPlot3D/}
    \item \textbf{Source Code:} \href{https://github.com/singhnitink/LigPlot3D}{https://github.com/singhnitink/LigPlot3D}
\end{itemize}

\section{Conflicts of Interest}
The author declares no conflict of interest.

\bibliography{references}

\end{document}